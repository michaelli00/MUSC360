\documentclass{article}
\usepackage[top=1in,bottom=1in]{geometry}
\usepackage{hyperref}
\usepackage{amsmath}
\usepackage{amssymb}
\usepackage{graphicx}
\graphicspath{ {./assets/} }
\usepackage[none]{hyphenat}
\date{}
\title{MUSC360 Music in Western Culture Before 1900: Music as a Public Art}
\begin{document} 
  \author{Michael Li}
  \title{MUSC360 Music in Western Culture Before 1900: Music as a Public Art}
  \maketitle
  \tableofcontents
  \newpage
  \section{Christimas in Paris, ca. 1280}
  During this time, Paris was one of Europe's largest cities and was a cultural and educational center.
  \begin{itemize}
    \item University of Paris (provided liberal arts education for white collar jobs)
    \item Cathedral of Notre Dame
  \end{itemize}
  The university and cathedral grew together with the literature culture, promoting poetry and debate. In particular, music was everywhere in Paris, especially amongst the aristocrats and church.
  \subsection{Church Music}
  Church music was performed in 2 main settings
  \begin{itemize}
    \item \textbf{The Office}: set of services in church conducted by and for the clergy (no laymen allowed).
    \item \textbf{The Mass}: daily service in church performed by professionals for laymen. Works were performed in Latin and included
      \begin{itemize}
        \item Spoken texts (e.g. readings and sermons)
        \item Sung texts (e.g. prayers and psalmody) set to \textbf{Gregorian chant}
      \end{itemize}
      Chants in particular can be broken down into 2 types:
      \begin{itemize}
        \item \textbf{Action chants} that accompany actions and rituals
        \item \textbf{Lesson chants} performed after readings to foster reflection of learned material.
      \end{itemize}
      Mass texts can be be broken down into 2 types:
      \begin{itemize}
        \item \textbf{Mass Ordinary}: texts performed every day
        \item \textbf{Mass Proper}: texts chosen according to the day of the liturgical calendar, usually from the \textbf{Book of Psalms} which has proper chants called \textbf{psalmody}.
      \end{itemize}
      Gregorian chants today are compiled in \textbf{Liber Usualis} and are notated using \textbf{square notation}.
  \end{itemize}
  \subsection{Puer natus est nobis (“A child is born to us”)}
  First chant of Christmas Day Mass. It is an \textbf{Introit}, which is an action chant\\
  It is proper and corresponding texts from Psalm 97 line 1 and the Book of Isaiah \\
  Performed as an \textbf{Antiphonal psalmody} (performed by 2 halves of the choir in alternation)
  \begin{itemize}
    \item \textbf{monophonic} in texture (only 1 line sung in unison)
    \item Pitch organization: \textbf{mode 7} (final - G, reciting tone - D). Reciting tone is the note that is repeated a lot.
    \item Melody is constructed from 
      \begin{itemize}
        \item Chanting on reciting tone
        \item \textbf{Melodic formulas} (melodic material shaped around punctuation and the grammar of the text).
      \end{itemize}
    \item Rhythmic organization is unknown.
  \end{itemize}
  Structure consists of
  \begin{itemize}
    \item Antiphon (opening segment that is based on \textbf{neumatic} text and multiple notes are sung on 1 syllable)
    \item Verse (marked Ps. and based on \textbf{syllabic} text where multiple syllables are sung on one note)
    \item Lesser Doxology (short repeated text that is required in every chant that they do, in the same tune as the verse and is \textbf{syllabic})
    \item Repeat of Antiphon (2nd choir marked by *)
  \end{itemize}
  \subsection{Viderunt omnes (“All the ends of the earth have seen the salvation of our God”)}
  \textbf{Gradual} (lesson chant performed after first reading) \\
  It is proper and corresponding text from text from Psalm 97, lines 2-4 (continues from the Introit) \\
  \textbf{Responsorial psalmody}: performed by a soloist alternating with a choir
  \begin{itemize}
    \item \textbf{monophonic} in texture.
    \item Pitch organization: \textbf{mode 5} (final - F, reciting tone - C)
    \item Melody constructed from
      \begin{itemize}
        \item multiple long \textbf{melismas}
        \item melodic formulas
      \end{itemize}
    \item Structure: 
    \begin{itemize}
      \item Response (opening by soloist and rest by choir). Material is \textbf{melismatic} (ornate and dozens of notes per syllable)
      \item Verse (marked V.) mostly soloist and choir enters at the end (entrance marked by *). Material is \textbf{melismatic}
    \end{itemize}
    \item Usually played as a musical highpoint of mass
  \end{itemize}
  \subsection{Types of Church Music}
  \subsubsection{Chants}
  \textbf{non-literate/unwritten} traditions where singers reconstruct melodies in performance using melodic formulas, performance conventions, and rules of Latin grammar.\\
  Important to note that this is neither memorized nor improvised. \\ \\
  Music notation was used for newly composed chant repertoire
  \begin{itemize}
    \item Sequences
    \item \textbf{Tropes}: textual or musical insertions into the body of the original chant and are usually sung by a solo voice. Used to elaborate on the theme of the day or explain something. \\
      Important to note that because this is adding to the original chant, tropes make the chant \textbf{proper}
  \end{itemize}
  \subsubsection{Organum}
  Earliest type of \textbf{polyphony} that consists of a 2-part texture 
  \begin{itemize}
    \item lower voice chant \textbf{(tenor)} sung by choir
    \item newly composed upper voice (\textbf{duplum}) sung by soloist
  \end{itemize}
  Typical organum practice during this time was \textbf{melismatic}
  \begin{itemize}
    \item slow, drone-like tenor
    \item fast, melismatic duplum
  \end{itemize}
  \textbf{Magnus Liber Organi} (“The Great Book of Organum”): collection of organum composed by \textbf{Master Leonin} and later revised by \textbf{Master Perotin}
  \begin{itemize}
    \item Above names were cited by \textbf{Anonymous IV}, a student (studied at \textbf{The School of Notre Dame}) from that era that named the composers and the works.
  \end{itemize}
  School of Notre-Dame brought a major innovation to music: \textbf{Rhythmic modes} system of rhythmic organization for organum for both composition and notation.
  \begin{itemize}
    \item notes are notated as either \textbf{long} or \textbf{breve} 
    \item Rhythmic patterns are grouped into \textbf{perfections} (units of 3)
  \end{itemize}
  Leonin's style, called \textbf{organum purum} ("pure organum") comprised of
  \begin{itemize}
    \item slow tenor, in free rhythm
    \item fast duplum in rhythmic modes
  \end{itemize}
  Perotin's style used \textbf{discants} (long melismas). In particular, when Perotin revised Magnus Liber, he replaced the organum purum with discants and composed new \textbf{discant clausulae} ("clauses"): long sections of discants that singers could substitute at will.
  \begin{itemize}
    \item fast tenor, in rhythmic modes
    \item fast duplum in rhythmic modes
  \end{itemize}
  \subsection{Leonin's and Perotin's setting of Viderunt omnes}
  \subsubsection{Leonin's Viderun omnes}
  Polyphony only seen in solo sections, choral sections are monophonic. \\
  The polyphonic portions can be split into 2 voices (tenor and duplum) and use organum purum style
  \subsubsection{Perotin's Viderun omnes}
  Consists of 4 voices that layer on top of each other, starting with the tenor (lowest)
  \begin{itemize}
    \item tenor
    \item duplum
    \item triplum
    \item quadruplum
  \end{itemize}
  Sections alternate between organum purum and discant clausulae. \\
  Upper voices share the same range and melodic ideas (\textbf{voice exchange}) \\ \\
  Vertical texture is a coincidence is a by product of the horizontal lines. Echo effect comes from the perfect consonances (4ths, 5ths, 8ves) \\
  During this time, 3rds and 6ths were seen as dissonant and thus were more hidden in the texture. Instead, 2nds and 7ths were more consonant and more prevalent in them music.
  \subsection{Secular Music and Types of Secular Music}
  Secular music was mostly seen in royal courts, festivals, parties.
  \subsubsection{Estampie}
  Popular dance at the time. 
  \begin{itemize}
    \item primary purpose of the music was to support the dancing so percussion instruments in particular were useful for counting steps.
    \item Monophonic tune was performed by an ensemble of loud instruments so that the music could be heard across the royal hall
    \item Structure consists of series of strains capped with open and then closed cadences
  \end{itemize}
  \subsubsection{Trouvère song}
  Monophonic secular song. Historically:
  \begin{itemize}
    \item Before ca. 1250: was used exclusively as courtly love poetry and songs for royalty and aristrocracy
    \item ca. 1250-1300: primary urban tradition of love poetry and songs where the royalty and \textbf{Puys} (brotherhood) sponsored performances
  \end{itemize}
  Genres include
  \begin{itemize}
    \item Chanson de jeste – a narrative epic
    \item Chanson courtoise – a song of courtly love
    \item Chanson de toile – a “woman’s song” (aka “spinning song”)
    \item Pastourelle – a pastoral
    \item Jeu-parti – a “mock debate”
  \end{itemize}
  \subsubsection{Motet}
  Most sophisticated "learned" secular genre that consists of 3 voices that are layered (tenor on bottom)
  \begin{itemize}
    \item Triplum (newly composed and based on French secular text)
    \item Duplum (newly composed and based on French secular text)
    \item Tenor (pre-existing untexted chant melisma)
  \end{itemize}
  \textbf{Adam de la Halle (d. ca. 1307)}: notable poet and composer who represented the last generation of trouvères. Was famous for both the poetic and musical aspects of trouvères (also famous for his motets). Normally only the poet is recognized and composer is not recognized.
  \subsection{Jeu de Robin et Marion (“play with songs”)}
  Trouvère song written by Adam de la Halle with a plot focuses on love story between Robin Hood and Maid Marion in a poetic/musical style. Song is organized into \textbf{refrains} (repeated with modifications) that make up a \textbf{rondeau} (round dance song). In particular, the structure is ABaabAB where lower cases are the refrains.
  \subsection{De ma dame vient/Dieus, comment porroie/Omnes}
  3-voice \textbf{Franconian motet} written by Adam de la Halle:
  \begin{itemize}
    \item 4 types of not durations that are grouped into perfections
    \item \textbf{polytextual}: duplum and triplum are set to different poems (representing the female lover and male lover that perform simultaneously but of sync) \\
  \end{itemize}
  Tenor part, which serves as a \textbf{cantus firmus} in particular is the opening melisma omnes from Gradual Viderunt omnes
  \begin{itemize}
    \item untexted and usually performed by an instrument, rather than voice
    \item slower than the upper 2 voices
  \end{itemize}
  Music style feature linear polyphony and the upper voices share the same range, with an emphasis on perfect consonances and accidental instances of dissonance.
  \subsection{Emergence of Polyphony}
  Hard to trace exact origins of polyphony but evidence suggests that early polyphonic music focused on amplifying monophonic chant, rather than serve as individual compositions. \\ \\
  Earliest recount of simultaneous sounding consonances are found in writings of \textbf{Hucbald} and \textbf{Regino} in the 9th century. The first writing of unambiguous polyphonic singing is in \textbf{Musica enchiradis} from the early 10th century. \\ \\
  Examples of polyphonic execution
  \begin{itemize}
    \item Wales: instrument execution was fast and lively 
    \item Britain: 2 part singing, 1 in a low register and the other singing something soothing to the ear at a higher range
  \end{itemize}
  Earliest polyphonic schools were centered around monasteries in France, England, Spain. It reached the School of Notre-Dame in the 12th/13th centuries and musicians from Paris were the first to notate both pitch and rhythm. \textbf{Anonymus IV} in his notes recounts several important figures during this time period. Most notable were
  \begin{itemize}
    \item \textbf{Master Leonius} known as best composer of organum
    \item \textbf{Master Perotin} best of composer of discant and introduced form of 4 voice writing (explained in previous section)
  \end{itemize}
 \end{document}
